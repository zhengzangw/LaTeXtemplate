\documentclass[language=english]{bilingualcv}
\begin{document}
\cvsetup{
    eauthor=Zangwei Zheng,
    eaddress={Jiangsu Province, Nanjing, Qixia District, Xianlin Road No.163},
    cauthor=郑奘巍,
    caddress=江苏省南京市栖霞区仙林大学城仙林大道163号南京大学仙林校区,
    mail=zzw@smail.nju.edu.cn,
    phone=(86)18852000199
}
\maketitle
\begin{education}
    \info{Nanjing University}[南京大学]{B.S Computer Science, Training plan of the national basic subject top-notch talent}[计算机科学与技术系,英才计划]{Nanjing, China}
    \begin{detail}
        \gparank{4.62/5.00}{3/31}{2017 -- Aug 2018}[2017 -- 2018.8]
        \comma[n]{Course Highlight}[课程节选]{ 
              {\it Probleam Solving \hspace{1cm}Introduction to Data Mining}
        }[\textit{问题求解\hspace{1cm} 数据挖掘导论}]
        \comma[n]{Academic Advisor}[学术导师]{\href{http://cs.nju.edu.cn/lim/}{Ming Li}, Ph. D., Professor}[\href{http://cs.nju.edu.cn/lim/}{黎铭教授}]
    \end{detail}
\end{education}
\begin{project}
    \info{\href{https://www.kaggle.com/c/nju--introdm/}{NJU Cloning Code Detecter}}[\href{https://www.kaggle.com/c/nju--introdm/}{克隆代码检测器}]{ML model to detect codes with the same function}[利用机器学习模型检测拥有相同功能的代码]{May 2018 -- June 2018}[2018.5 -- 2018.6]
    \begin{detail}
        \entry{Generated Matrix Representation for each code and created new formula\\ for Judgement Complexity}[将代码转换为表示矩阵,设计新的条件复杂度公式]
        \entry{Implemented Random Forest with Numpy and compared with many\\ other models}[使用随机森林及其它学习算法设计模型]
    \end{detail}
    
    \infot{\href{https://github.com/zhengzangw/NJU\_MIPS}{NJU\_MIPS}{A MIPS32 system with 5 levels pipeline CPU}}[\href{https://github.com/zhengzangw/NJU\_MIPS}{五级流水线 MIPS32 系统}]{Dec 2018 -- Jan 2019}[2018.12 -- 2018.1]

    \begin{detail}
        \entry{Finish nearly all mips instructions alone}[利用 Verilog 基本独立完成五级流水线 MIPS 系统的所有指令]
        \entry{Connected with keyboard, monitor and digital piano with MMIO}[连接键盘、显示器和电子琴等外设并编写驱动和内存地址映射]
        \entry{Support C program with Input/Output}[利用运行时环境(AM)编译C程序到CPU上运行]
    \end{detail} 

    \backup{
        \info{Restaurant Visitor Predictor}{Predict how many future visitors a restaurant will receive for {Kaggle Competition}}{Nov 2017 -- Jan 2018}
        \begin{detail}
            \item Collected relevant data and cleaned data with Time Series Analysis
            \item Implemented RNN with Tensorflow
        \end{detail}
    }
\end{project}
\begin{skill}
    \ecomma[b]{Proficient}{C/C++, Python}
    \ecomma[b]{Familiar}{Linux, Java, Verilog, \LaTeX}
    \ecomma[b]{Tools}{Git, Matlab, Adobe Photoshop/Premiere}
\end{skill}
\begin{leadership}
    \info{Nanjing University}[南京大学]{Leader of 2017 Computer Science top-notch talent class}[2017级计算机拔尖班团支书]{Sept 2017 -- Present}[2017.9 至今]
    \begin{detail}
        \entry{Organized two activities each month for class students}[每月组织两场班级活动,如:知识讲座、志愿活动、学科交流]
        \entry{School May 4 outstanding cadres}[五四优秀团支书][May 2018][2018.5]
    \end{detail}
\end{leadership}
\begin{activity}
    \info{Debate Player}[辩论选手]{Debate Player of Computer Sicence}[计算机系辩队队员]{Oct 2017 -- Oct 2018}[2017.10 -- 2018.10]
    \begin{detail}
        \entry{One of the Best Debater in College Cup, 15/80}[院系杯约15位最佳辩手之一]
    \end{detail}
\end{activity}
\begin{honor}
    \info{ACM--ICPC Xuzhou National Invitational Tournament}[ACM--ICPC 徐州邀请赛]{Silver Award, 20\% of 195 teams}[银奖,195支队伍中前20\%]{June 2018}[2018.6]
    \info{The Second Prize Scholar Ship of top-notch Talent Students}[拔尖计划二等奖学金]{25\% of students in training plan of the national basic subject top-notch talent}[拔尖计划学生中排名前25\%]{Sept 2018}[2018.9]
\end{honor}
\makedate

\end{document}